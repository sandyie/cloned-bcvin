\documentclass[11pt,letter]{article}
\usepackage[top=1.00in, bottom=1.0in, left=1.1in, right=1.1in]{geometry}
\renewcommand{\baselinestretch}{1.1}
\usepackage{graphicx}
\usepackage{natbib}
\usepackage{mathtools}
\usepackage{gensymb}
\usepackage{hyperref}

\def\labelitemi{--}
\parindent=0pt

\newenvironment{smitemize}{
\begin{itemize}
  \setlength{\itemsep}{0pt}
  \setlength{\parskip}{0.8pt}
  \setlength{\parsep}{0pt}}
{\end{itemize}
}

\begin{document}

\title{Thoughts on an improved temperature model}
\author{Lizzie, so far}
\date{\today}
\maketitle
\tableofcontents

\newpage
\section{Plant temperature}

Continuing our discussion of how to improve our temperature model, I did some reading in grapevine canopy and berry temperature, especially compared to the temperature data that we have---measured air temperature. \\

This is a well-studied topic. People have been using basic energy flux models for this since at least \citet{millar1972} and developing them here and there into models growers can use to estimate berry temperature \citep[e.g.,][]{cola2009}. There are also continuing comparisons of how canopy and berry temperatures relate to measured air temperature \citep[e.g.,][]{costa2019,pena2020}. After all those years, though I am not sure there is an agreed-upon model (I found the \citet{millar1972} the best paper I read). Papers from Washington state diverged from those in Chile, and these's not even consistent info on `our temperature station was $x$ far away and in this type of location,' so I feel a little uncomfortable doing too much more than thinking on where and how to let the plant temperature vary from the air temperature. \\

That said, I think I gleaned a few insights useful to our priors and our temperature model. 

\begin{smitemize}
\item Sun and shade matter to berries: \citet{millar1972} found exposed fruit was 1.4-7.3 C higher than measured air temperature and shaded fruit were 0.5-4.4 C lower. (Relatedly, most radiative heat for berries goes into sensible, so you can ignore latent heat etc.)
\item Sun and shade matter to leaves also: \citet{millar1972} found exposed leaves were similar to measured air temperature and shaded leaves were 0.9 C lower. (Relatedly, most latent heat matters a lot with leaves.) \citet{pena2020} found canopy seemed to increase diurnal temperature range (canopy minima 1.2 C less than air and maxima 2.0 higher) though I am not sure how consistent this is as other studies did not mention it. 
\item Aspect and row positioning, as well as trellising matter to temperature \citep{costa2019}, though I didn't find a consistent model of this. But it makes sense.
\item From conversations with collaborators---cover crop matters, so if you have a cover crop between rows that will reduce temperature.... at least while it's alive. So does irrigation. And irrigation x cover crop: if you have overhead irrigation it keeps your cover crop alive longer and keeps temperatures lower.
\item Lags: I found conflicting info on whether plant temperature lags behind air temperature (or vice versa I guess). \citet{costa2019} found a lag of several hours for max temp comparing canopy (later max, around 17:00) and air (high from 11-17:00) while \citet{pena2020} reported no lag.
\end{smitemize}

Here's how this could matter to our model:

\begin{smitemize}
\item Several of these factors are vineyard-specific (trellising, irrigation, generally cover crop) and some are block-specific (aspect, row positioning), suggesting we might want to let the temperature model vary at these levels if possible.
\item Some of these factors vary across the season: the cover crop often dies later in the season and growers often remove leaves to increase light/warmth in cool climates later in the season. But then, other places leave the leaves up .... not sure we want to deal with this. 
\item We often assume 40 C is the upper limit for most plant processes, as proteins denature above that, but we might want to think about where to set that prior (i.e., make sure it is on plant-temperature, not air temperature).
\item I think we can assume no lag, since there is not a known one. 
\end{smitemize}

% How much of this should be embedded in the model? Growers want models that work with air temperature so ....


\section{Lizzie's early-August thoughts}

I think the sine curve is not so bad. If we want to improve we could consider a spline (or something similar) {\bf to capture weather fronts,} which is our main issue (to me, at least, from looking at the residuals). I would also like to {\bf add a baseline long-term trend} since most locations are getting warmer with climate change. Even if we don't see it here I'd be interested in having it in the model soon-ish as we would want it in most models. 

\section{Overview of options considered}

\begin{enumerate}
\item Fourier: I love Fourier, but I am not sure what we'll get from this beyond the sine curve for climate data here.  
\item Autoregressive: Maybe, but I worry these will require a lot of decisions we're not sure of---the lag may vary over time, the timing of seasons varies ... they feel not flexible enough to capture weather fronts. 
\item Sine curve: Seems pretty good as a first pass, but looks to miss weather fronts. 
\item Splines: this is just a catch-all term for smoothing f(x)s. They seem to usually chop up the time-series to create some bends. I have seen a lot of climate folks use thin-plate splines, but I am still trying to figure out why. 
\item GAMs: This is just a generalized additive model where the linear predictor uses some smoothing f(x)s. 
\item ... 
\end{enumerate}

\section{Emails and other notes}

\subsection{Christy Rollinson email (July 2020)}

With the temperature modeling are you predicting past or future?  I have a couple approaches I've used for gap filling and spaitio-temporally downscaling daily and subfamily meteorology data with uncertainty, but they're probably overkill.  I've been using splines in GAMs a lot for this because of the nice tradeoff between being fairly computationally quick, ability to capture ``surprises''  and a variety of shapes among sites with the same function, and ability to robustly estimate uncertainty.\\

\subsection{Ben Cook email (July 2020)}

As for your query, I guess it depends on exactly what you want to to do. By ``model the temperature data'', do you mean you want to generate a new synthetic time series (say, daily data for one year) sampled from the existing data? One approach to do this would be to, using the multiple years of data you have, generate an average daily climatology, and. then convert each year to ANOMALIES. Once could then resample from the anomalies (e.g., using an autoregressive approach) and then superimpose these new synthetic anomalies back on to the climatology. You would probably want to fit a different autoregressive model to each season (DJF, MAM, JJA, SON). \\

\subsection{From Ailene (July 2020)}
Lizzie chatted with Ailene in late July. Ailene was very pro-splines and just sent some links.\\

Splines are the heart of the brms non-linear modeling (see \href{https://mran.microsoft.com/snapshot/2017-05-14/web/packages/brms/vignettes/brms_multilevel.pdf}{here}).\\

\href{https://fromthebottomoftheheap.net/2018/04/21/fitting-gams-with-brms/}{Link about GAMs} (in brms).\\

\href{https://github.com/milkha/Splines_in_Stan/blob/master/splines_in_stan.pdf}{Splines in Stan.}\\

\subsection{Random references from Lizzie}

\href{https://www.fs.fed.us/rm/pubs/rmrs_gtr165.pdf}{A Spline Model of Climate for the
Western United States} for monthly data, from 2006\\

\href{https://rmets.onlinelibrary.wiley.com/doi/pdf/10.1002/joc.4068}{For daily rainfall} but seems to have a good review. \\

\href{https://agupubs.onlinelibrary.wiley.com/doi/full/10.1002/2013JD020803}{Another article with a semi-useful intro}.\\

\href{https://www.nature.com/articles/sdata2018299}{A 2019 thin-plate spline paper}

\subsection{From I\~naki email (July 2020)}

The only paper I can propose to read is the chapter we wrote with Isabelle about phenology models (see attached file). There is an important review of different models used to simulate phenology stages  but it was quite focused on dormancy (endo and eco) phases. 
There is also this one from Rebaudo et al. (focused on insects).\\

Usually we tested last years (with Amber and other colleagues): GDD, Sigmoid, Richardson, Wang, Normal (also called Chuine), triangle. 
There are other interesting as Logan (1, 2, 3), Anderson, Sine. But I did not tested them for plants and they are very similar to other models. \\

Lizzie notes: the two files are in the phenmodels/temperatre refs folder, but I am not sure they include temperature models.

\bibliographystyle{..//..//..//refs/styles/amnat}
\renewcommand{\refname}{\CHead{}}

\emph{References}
\bibliography{..//..//..//refs/planttemperature/plantemp.bib}

\end{document}

https://gracilis.carleton.ca/davidson/publications/2010/Newlands_Environ_2010.pdf


