\documentclass[11pt,letter]{article}
\usepackage[top=1.00in, bottom=1.0in, left=1.1in, right=1.1in]{geometry}
\renewcommand{\baselinestretch}{1.1}
\usepackage{graphicx}
\usepackage{natbib}
\usepackage{amsmath}
\usepackage{url}

\def\labelitemi{--}
\parindent=0pt

\begin{document}
\bibliographystyle{/Users/Lizzie/Documents/EndnoteRelated/Bibtex/styles/besjournals}
\renewcommand{\refname}{\CHead{}}

{\bf Notes on winegrape hardiness work}\\

Last udpated: \today\\

\emph{Project aim:} Build hierarchical models of winegrape (\emph{Vitis vinifera} subsp. \emph{vinifera}) hardiness that best predict hardiness when compared to other major modeling options.

\section{Work done to date}
Carl Bogdanoff (AgCanada) has been collecting hardiness data for 7 years across several varieties and several vineyards (super cool data!) and has fit a model of hardiness for Chardonnay (he worked on this for several years) and now a model of Merlot (just completed, following his general design of the Chardonnay models). Since the summer Lizzie has \emph{slowly} been trying to translate the models from Excel to R (see \path{..\\analyses\hardiness_stab.R}). \\

Meanwhile other labs have worked on this too. We should check out work by Londo (led mostly, I believe, by his former student Kovaleski) and Markus Keller's group. Also, check out the French group of Charrier. 

\section{Next steps}

\begin{enumerate}
\item Finish translating Carl's models from Excel to R (keep Carl and Lizzie in the loop).
\item If possible, code up models from Kovaleski work and Markus Keller (and Charrier if there is a published model), then compare all THREE modelling approaches using comparison metrics {\bf including cross-validation}.
\item Code up a hierarchical modeling approach that includes the basics for hardiness but allows variation across varieties and vineyards. 
\item If all goes well, project to future!
\end{enumerate}

\end{document}