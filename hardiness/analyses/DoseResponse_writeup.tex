%document describing teh final version of ny dosse response model and what I think about its fit. Writen by Faith, started Augst 21st 2020.
\documentclass[11pt,letter]{article}
\usepackage[top=1.00in, bottom=1.0in, left=1.1in, right=1.1in]{geometry}
\renewcommand{\baselinestretch}{1.1}
\usepackage{graphicx}
\usepackage{natbib}
\usepackage{amsmath}
\usepackage{textcomp}%amoung other things, it allows degrees C to be added
\usepackage{float}
\usepackage{hyperref}
\usepackage[utf8]{inputenc} % allow funny letters in citaions 
\usepackage[nottoc]{tocbibind} %should add Refences to the table of contents?
\usepackage{amsmath} % making nice equations 

\def\labelitemi{--}
\parindent=0pt

\title{Winegrape Hardiness \\ My final (I hope) dose response model}
\date{\today}
\author{Faith Jones}


\begin{document}
%\renewcommand{\refname}{\CHead{}}%not sure what this was supposed to do 
\renewcommand{\bibname}{References}%names reference list 

{}
\maketitle{}
\tableofcontents

\section{Background}
Dose response curves are regression models. The exact form of dose response models varies somewhat, encompassing a range of different statistical models including nonlinear regression models, generalized regression and parametric survival analyses \cite{Ritz2015}. What connects these models is their application: to model the effect of some dose on a biological response. The model assesses the relationship between what is traditionally named the ``dose'' or ``concentration'' and the ``response'' or ``effect''. This terminology comes from the model's  original use in pharmacology for modeling the effects of substances on physiology. The dose is usually some sort of biological stress that elicits a response from a biological organism. Dose values should be non-negative \cite{Rudemo1989} and the response values should change monotonically with increasing dose. Variations on the dose-response model have been applied to a wide variety of ecological questions, including species richness responses to nitrogen \cite{Jones2018a}, eco-toxicology \cite{Haanstra1985}, and phenology (winegrape budburst dates) \cite{Kovaleski2019}. \\

My analyses focus on sigmoid curve log-logistic models. These models are used to model relationships between variables with asymptotic minimum and maximum response values. I use the four parameter log-logistic model described in \cite{Ritz2015} and the below section, where the relationship between the dose and the response is a function of a maximum response, a minimum response, a response rate, and a dose giving a 50\% response. A benefit of this model is that the parameters are easily interpretable and biologically meaningful \cite{Seefeldt2016}\\

\section{Model Structure}
\subsection {Basic Model}

As introduced above, I use a four parameter log-logistic model to model winegrape cold hardiness Equation \eqref{eq:1}. We used mean 2 day air temperature for our modeling because \textbf{Check Carl's reasoning} and because winegrape hardiness is closely linked to mean air temperatures \cite{Hubackova1996}. Air temperature readings were taken at \textbf{Penticton agCanada site??? check}. \\

Some modifications to the original data are needed, though, before analysis. Firstly, to ensure the dose values are always positive, I added 30 to each air temperature value. Secondly, I multiplied the hardiness values with \-1 so that larger values in the model meant higher winter hardiness. This was to make interpreting the minimum and maximum asymptotes of the model more intuitive. \\

\begin{equation*}
\label{eq:1}
\mu=f(x,(b,c,d,e))=c+\frac{d-c}{1+exp^{b(log(x)-\tilde{e})}}
\end{equation*}
\begin{equation*}
\label{eq:2}
\tilde{y}_{i}\sim normal(\mu_{i},\sigma)
\end{equation*}

Where:\\
	$x$ is the concentration of the dose (amount of winter cold) \\
	$b$ is the response rate (slope)\\
	$d$ is the upper asymptote of the response (maximum hardiness)\\
	$c$ is the lower asymptote of the response (minimum hardiness)\\
	$e$ is the effective dose ED50 (winter temperature where cold hardiness is half way between min and max)  \\
	$\tilde{e}$ is the log of the effective dose ED50\\

\subsection {Full Model}

The final model includes hierarchical variance for different varieties and sites on the $d$ (maximum hardiness) parameter and for variety on the $b$ (rate of change) parameter Equation \ref{eqn:modelWithPriors} \\

We expected different sites to vary in their maximum hardiness because the model uses temperature data for a single site but the weather conditions at sites in the Okanagan Valley can vary substantially. For example (\textbf{insert name of site}) is a site on a south facing slope close to the lake shore and so is warmer, whereas the colder site of (\textbf{insert name of site}) is further north and more inland. Such micro-climatic differences should cause maximum hardiness to be less in warmer sites and more in colder sites. \\

Winegrapes have been domesticated for many thousands of years, and over that time growers have cultivated a wide range of varieties (genetically unique variants) with different physiological and ecological characteristics. Winegrape varieties consequently vary a lot in many of their traits. Although the exact mechanisms behind winegrape winter hardiness are unknown, winter hardiness seems to vary across varieties \cite{Mills2006,Ferguson2014,Kovaleski2018a}. Winegrapes may have different rates of change of winter hardiness\cite{Kovaleski2018a,Ferguson2014} and different maximum hardiness values \cite{Ferguson2014}. We included a hierarchical effect of variety on both maximum hardiness and the rate of change of hardiness so we could assess how variable variety specific winter hardiness is. \\

We built our dose response model in a Bayesian framework using Stan \textbf{cite stan version} in R \textbf{cite rstan and r}. An essential part of modeling using Bayesian methods is the choice of prior expectations on each parameter value. Our priors are specified in Equation \ref{modelWithPriors}, and were generally chosen to encompass all possible parameter combinations according to our current physiological understanding of winegrape hardiness. The exception to this is parameter $c$, minimum winter hardiness. Our data did not span the full range of the sigmoid curve relationship of winter hardiness to air temperature; we lack data on minimum hardiness. This is a problem for model estimating. An estimation of minimum hardiness of winegrapes was taken from a selection of sources: -3\textdegree C \cite{Ferguson2011}, -1.2\textdegree C \cite{Ferguson2014} \textbf{more sources}. We fed this estimation into the model as a prior constrained closely around -2\textdegree C.  





\begin{equation*}
\label{modelWithPriors}
\mu=f(x_{i},(b,c,d,e))=c+\frac{(d+d_{var,i} + d_{site,i}) -c}{1+exp^{b_{var}(log(x_{i})-\tilde{e})}}
\end{equation*}
\begin{equation*}
{d}_{var} = dr_{var} * \sigma_{dvar}
\end{equation*}
\begin{equation*}
{d}_{site} = dr_{site} * \sigma_{dsite}
\end{equation*}
\begin{equation*}
{b}_{var} = br_{var} * \sigma_{bvar}
\end{equation*}
\begin{equation*}
\tilde{y}_{i}\sim normal(\mu_{i},\sigma)
\end{equation*}

Where:\\
	$x$ is the concentration of the dose (amount of winter cold) \\
	$b$ is the response rate (slope)\\
	$d$ is the grand upper asymptote of the response (maximum hardiness hardiness) 
	$d_{var}$ is the effect of each variety on the upper asymptote of the response (maximum hardiness)\\
	$d_{site}$ is the effect of each site on the upper asymptote of the response (maximum hardiness)\\
	$\sigma_{dvar}$ is the standard deviation of the effect of varieties on maximum winter hardiness\\
	$\sigma_{dsite}$ is the standard deviation of the effect of sites on maximum winter hardiness\\
	$dr_{var}$ is the non centred parameterization values for varieties effect on maximum hardiness d\\
	$dr_{site}$ is the non centred parameterization values for sites effect on maximum hardiness d\\
	$b$ is the lower asymptote of the response (minimum hardiness)\\
	$\sigma_{bvar}$ is the standard deviation of the effect of varieties on rate of change of winter hardiness\\
	$br_{var}$ is the non centred parameterization values for varieties effect on rate of change\\
	$e$ is the effective dose ED50 (winter temperature where cold hardiness is half way between min and max)  \\
	$\tilde{e}$ is the log of the effective dose ED50\\

Priors (hardiness has been multiplied with -1 to be positive, and 30 has been added to air temp)\\
	$b \sim gamma(7,1)$\\
	$\sigma_{bvar} \sim normal(0,3)$ \\
	$br_{var} \sim normal(0,1)$ \\
	$d \sim Normal(25, 10)$ \\
	$\sigma_{dvar} \sim gamma(2.5,1.75)$ \\
	$dr_{var} \sim normal(0,1)$ \\
	$\sigma_{dsite} \sim gamma(2.5,1.75)$ \\
	$dr_{site} \sim normal(0,1)$ \\
	$c \sim normal(2,0.5)$\\
	$\tilde{e} \sim normal(log(30), 0.15)$ \\
	$\sigma \sim normal(0,5)$\\



\section{Model fit}
\subsection{Prior Predictive Checks}
Plot of predicted values
\subsection{Retrodictive Checks}
Pairs plot
Plot of predicted values against real data
plot of predicted data against real data

\subsection{Estimated Parameter Values}
Table of values for each parameter 
Panel of plots for each parameter 
Plot of effect of variety and site on d
Plot of effect of varieties on d
Plot of effect of sites on d
Plot of effects of varieties on b

\subsection{Predictions}
Description and citation of data from Washington 
Plot of model prediction x ~ y 


\section{Discussion}

Generally the model seems to do a good job of predicting values, even from new datasets/areas. 

Some uncertainty around predicted values though 

Model overestimates hardiness in autumn if there is an especially cold snap - how can I work with this?

Site differences - how do these compare to our (Carl's) knowledge of the geography of the sites?

Variety differences - compare to results in \cite{Ferguson2014}'s table of winegrape variety max hardiness. Also contrast our results of no effect on rates of change to results by \cite{Ferguson2014,Kovaleski2018a,Kovaleski2019}. 

Compare relative importance of variety and site for growers adapting to climate change. 

\bibliographystyle{/home/faith/Documents/Bibtex/styles/besjournals/besjournals}



\bibliography{/home/faith/Documents/Bibtex/mendeleyStuff/library}

\end{document}

