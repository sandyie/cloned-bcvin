%document describing teh final version of ny dosse response model and what I think about its fit. Writen by Faith, started Augst 21st 2020.
\documentclass[11pt,letter]{article}
\usepackage[top=1.00in, bottom=1.0in, left=1.1in, right=1.1in]{geometry}
\renewcommand{\baselinestretch}{1.1}
\usepackage{graphicx}
\usepackage{natbib}
\usepackage{amsmath}
\usepackage{textcomp}%amoung other things, it allows degrees C to be added
\usepackage{float}
\usepackage{hyperref}
\usepackage[utf8]{inputenc} % allow funny letters in citaions 
\usepackage[nottoc]{tocbibind} %should add Refences to the table of contents?
\usepackage{amsmath} % making nice equations 

\def\labelitemi{--}
\parindent=0pt

\title{Carl Faith meeting 23rd Sep 2020}
\date{\today}
\author{Faith Jones}



\begin{document}
%\renewcommand{\refname}{\CHead{}}%not sure what this was supposed to do 
\renewcommand{\bibname}{References}%names reference list 

\section{Overview}
\begin{itemize}
	\item Carl is exited about the dashboard and had suggestions on useful info/tools to include
	\item The results of my model make sense in terms of variation between variety and site, but the model needs day of year (or something similar) to be useful to growers
\end{itemize}

\section{Temperature change in Okanagan Winters}

I spoke to Carl about our idea for a manuscript that looks at how the minimum winter temperatures have changed in the Okanagan in relation to modeled maximum hardiness from different varieties. Carl said he has done something similar as a newsletter that he will send me. He looked at change in average temps over the last 100 years and they are 3 degrees warmer now. Also fewer ``killing winters'' were temps drop below -23\textdegree C. He pointed out that maximum hardiness wont change much each year. \\

\section{Comments on the dose response model}

Carl was a bit confused/disappointed that day of year was not in the model, although he liked the simplicity of my dose response model. We discussed how growers are most interested in vine hardiness during deacclimation and acclimation because this is where they are often most at risk, and this is difficult to get from my model. I agreed, and said we are hoping to tackle that somehow. Carl with give it some thought. \\

Carl also suggested we contact Jim Willworth from Brock for more hardiness data, and liked the idea of a model that works for different regions. We talked about the problems of the current models being specific to sites or regions. Carl would like us to try his model with other hardiness data to see how and where it fails because this might be a useful exercise. He suggested maybe consider day length or try and tie in phenology? Time of year is is very important so I should try and include it somehow. \\

three degrees is a huge difference to grapes, so the effect of site and variety is not trivial. The more southern sites are generally colder. ALso vicinity to the lakes matters. Oliver east is higher up so tends to be colder. Carl actually has weather data for each site, he just decided to use Penticton weather data. Its difficult with Covid, but he will try and get that data so I can use it in the model. This would be cool because we can then separate the effect of site (i.e. management decisions and variety clone) from the effect of climactic variations. \\


\section{Dashboard}

Carl is exited by the project, and is looking forward to having something to beta test. Ideally the model should update every day because growers need data about what the vines are doing in real time. \\

We discussed using data from different weather stations. Carl thinks it is a good idea but we need to check the model doesn't do anything odd. He also suggest that we produce a table of hardiness values (for each variety?) for plus and minus degrees because growers should know how much warmer or colder their sites are than Penticton. Also it would be great if there was a tool letting growers input their own data because growers tend to have their own weather stations. But again need to check model is not too sensitive! \\

Someone needs to keep collecting hardiness data so we make sure the model is still accurate. Who will do that? Ag Canada? or Industry? Car is trying to find the most ``representative'' site to be monitored. \\

\section{Other comments}
Carl is fine with us using his data for a big collaborative projects because ``I'm retiring''. \\

We have agreed to tentatively touch base every month, and I will show him the dashboard when there is something to see. 


\end{document}