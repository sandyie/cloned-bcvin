\documentclass[11pt,letter]{article}
\usepackage[top=1.00in, bottom=1.0in, left=1.1in, right=1.1in]{geometry}
\renewcommand{\baselinestretch}{1.1}
\usepackage{graphicx}
\usepackage{natbib}
\usepackage{amsmath}
\usepackage{textcomp}%amoung other things, it allows degrees C to be added
\usepackage{float}
\usepackage{hyperref}
\usepackage[utf8]{inputenc} % allow funny letters in citaions 
\usepackage[nottoc]{tocbibind} %should add Refences to the table of contents?
\usepackage{amsmath} % making nice equations 

\def\labelitemi{--}
\parindent=0pt

\title{Winegrape research \\ Ideas from reading the literature }
\date{\today}
\author{Faith Jones}


\begin{document}
%\renewcommand{\refname}{\CHead{}}%not sure what this was supposed to do 
\renewcommand{\bibname}{References}%names reference list 


\maketitle
\tableofcontents

\section{Miscellaneous}

Winegrape vines take 4-5 years to reach reproductive maturity, and can remain economically  productive for 50-60 years (common knowledge? See \cite{Bindi1996}. So planting varieties that will be able to adapt to predicted climate in the future is important. Not much scope for quick turnaround of varieties, need planning in advance.\\

Historically, winegrape growers have adaptively made changes to viticultural practices in light of climate change \citep{Ashenfelter2016}– economic. \\

\cite{VanLeeuwen2017} – a good general overview of how viticulture can adapt to climate change, mostly by getting vines to veraise later. \\

vines will veraise 16-24 days earlier by the end of the century in comparison to now \citep{Duchene2010} (for GW and RI in France). Also the biggest changes are predicted to be in the first half of the century. \\

\cite{Ferguson2014} estimated day of bud-burst, initial hardiness and maximum hardiness were all positively correlated. More hardy varieties in winter tended to budburst earlier in the spring. \\

If I have enough data, I would like to look at the variation due to root stock and clone. What capacity is there for retaining a variety but still adaptive planting?\\

The ideal window for harvesting grapes in the northern hemisphere is 10th September to 10th October. Because of climate change some wines like merlot and Sav blanc wont be so high quality in Bordeaux because their picking dates earlier than 10th September. See \cite{VanLeeuwen2019}. How do we adapt to this?
\begin{enumerate} 
\item Plant later ripening varieties (but European regulations and market demand)
\item Plant later ripening clones
\item Change rootstock. Data is scarce though.		
\item Increase Trunk height so vines don't get as warm in the summer (how does this affect their winter hardiness?). \cite{VanLeeuwen2019,VanLeeuwen2017} talk about a special goblet growing shape that helps, but you cant mechanically harvest.
\item Reduce leaf area to fruit weight ratio. Fewer leaves mean less sugar accumulation in the 	berries during verasion. More applicable for white wine than red
\item Late pruning in winter delays budburst in spring by a few days. Still quite experimental 	though. 
\end{enumerate}

Could any of the viticultural practices employed to encourage later ripening also affect the vines ability to maintain winter hardiness? Perhaps a problem for places like the Okanagan where summer temperatures are already high. \\

Increased drought likely because warmer and more evapotransporation. This can have positive or negative effects on wine quality, depending a lot on how extreme the conditional already were \citep{VanLeeuwen2019}. \\

How much clonal variation is there in winegrape winter hardiness? There is quite a bit for sugar accumulation dynamics \citep{VanLeeuwen2019} (fig 8) and sugar accumulation plays an important role in hardiness.  \\

what about clones? Choosing the right clone is a powerful tool to deal with drought stress (V), so maybe for winter hardiness too?\\

Blueberry varieties that were more cold hardy also tended to require more chilling \citep{Arora2004}.\\

Is there a relationship between maximum winter hardiness and ideal growing temperatures? See \cite{Ashenfelter2016} for a plot of variety optimal temp for growing. \\

There is some capacity for breeding more appropriate varieties, but it will only help so much. \cite{Duchene2010} found that crossing varieties was insufficient to breed varieties in Alsace that would reproduce wine characteristics into the second half of the 21st century. Maye better to switch varieties altogether? Also this paper explains nicely the problem of optimum ripening temperatures shifting later in the season but verasion dates shifting earlier due to warmer temperatures. \\ 

Changes in the duration of periods between flowering and verasion in Riesling in Alsace \citep{Duchene2005} no clear change for budburst-flowering. Also dry summers may cause problems. \\

\cite{Hannah2013} looked at projections of where will be suitable for winegrapes considering climate change. They predict that the area suitable will change a lot, and this will probably cause conflicts with biodiversity/ecological conservation. \\

\section{Climate Modeling}

Good wine making is very sensitive to climate, so much so that wine was described as the "canary in the coal mine for climate change" \citep{Goode2012}.

The sensitivity of wine making to climate change threatens natural systems through potential land conversion and increased irrigation \citep{Hannah2013}.

I could frame a question around predicting species’/variety's niches? This might be more ecological? About how difficult it is based on a one or a few climate variables. Focus on supposed increases in where wine can be grown in the north because winter temperatures are getting higher, but what about the influence of increasing heatwaves, drought and false springs? Also how does considering within species variation change the answer? But what if better adapted varieties cant reach new areas? Also how much can we infer about changes in summer heat/drought from changes in winter temperatures? Do places generally increase winter and summer temperatures the same amount? Does something weird happen in spring? Might there be more extreme fluctuations in climate? Are microclimatic differences consistent across seasons? \\

Are more cold tolerant species more or less sensitive to heat/drought? Maybe some trade off? This will matter when considering new areas to grow grapes. \\

What about the potential affects of lack of chilling in warmer winegrape regions? See \cite{Luedeling2009} about this problem in trees. \\ 

Could vineyards use micro climates within regions to find the perfect balance of warmer winters without too hot summers? Maybe a slope aspect or something? I could take a look at this by seeing if there are any small scale climate patterns in which areas will be most suitable. I.e. mostly a particular facing slope or something. \\

\cite{Fraga2016a} overview of projected changes in climate (and other stuff) and how this will change viticulture in Europe. More yield in France. Can grow grapes in UK. Spain will have lower yield. Changes in phenophases. Also strong regional heterogeneity. \\ 

How should I deal with climate modeling uncertainty? See \cite{Deser2012}. \\

weather stations local but outside the vineyard don't give perfect results – they can overestimate or underestimate max and min daily temperatures \citep{Antivilo2017}. \\

\cite{Blanco-Ward2017} analysis climate change and Portuguese winegrapes. They focused on the number of summer days with Tmax$>$25\textdegree C, the number of very hot stressful days with Tmax $>$35\textdegree C, CSDI-cool spell duration index, WSDI – warm spell duration index, R10 – number of days with heavy precipitation (daily $>$ 10mm), CED – maximum consecutive wet days $>$1mm. \\

\cite{Blanco-Ward2017}'s modeling suggests significant advancing verasion (22days) without noticeable change in variance. Also more days that are too hot, and greater variation in this.  \\

Small spatial extents can have very different climates \citep{Courault2001}. \\

the heat requirements of different phenological stages were independent across varieties \citep{Duchene2010}. So a variety that only needs a little warming in the spring to burst bud may need a lot of warming to flower or veraison? Does this mean selecting the optimum variety will be more complicated? Are there varieties that are equally cold hardy in winter but have quite different requirements for summer warmth to veraise? \\

Winkler Index – bioclimactic index to classify winegrowing regions based on the thermal requirements of winegrapes. Calculated from the sum (between April and October) of the daily average temperature  exceeding 10 degrees c. How well does this index describe the climactic variability within the valley in Washington or the Okenagan valley? \\

In the Mediterranean basin, spring and summer are predicted to warm more than autumn and winter \citep{Ferrise2016}. Is that the case in my datasets too? Does this mean that enough warming in winter to decrease frost damage will mean too much increase in temperatures in summer so a bad quality crop? Also temperatures increases are patchy. Also entire grapevine growth cycle shortened between 6-10 days on average. This is a problem because shorter time to grow means less yield of grapes. \\

Need to consider the effect of higher CO$^2$ alongside warming conditions. \\

Winegrape hardiness is closely linked to daily maximum and mean temperatures, rather than minimum temperatures \citep{Hubackova1996}. 

\subsection{List of variables to look at}

\begin{itemize} 
\item Number of days (consecutive?) below -x temperature. What temperature? To get a general feel for how cold the winter is
\item Number of days below a really cold temp – maybe below maximum hardiness of Riesling, or whatever grape variety is most commonly grown. Or below -25\textdegree C?
\item Number of days above 40\textdegree C (or above 35\textdegree C see \cite{Blanco-Ward2017})
\item Number of consecutive days above a high temperature.
\item Average temperature during verasion period? 
\item Sum of daily average temperature  exceeding 10\textdegree C between April and October (Winkler)
\item Temperature variability annually – assume the variability is not increasing over time, but instead mean is increasing. Are there areas where variation is increasing or decreasing?
\item Some measure to capture false spring events? Maybe these are not much of a problem in Napa? 
\item Number of soil-freezing days during growing and harvest season.
\item Numbers of sunny hours? Because sunlight is so important for grapevine quality. 
\end{itemize}

\section{False Springs} 
Could I get at which varieties are going to be less vulnerable to false springs? \\

How much of a problem will false springs be? Will they be more of a problem for areas that are only just getting warm enough for planting? Will this cause problems if you need to plant a very hardy vine but then it is more susceptible to false springs? $**need more research on false spring modeling**$
Most damage to crops occurs during spring frosts or autumn heat waves \citep{Charrier2013}. So this is where my model need s to be most accurate? \\

Phenology - Do I see evidence that more cold hardy varieties budburst earlier? See \cite{Ferguson2014}. If so these varieties may be \textbf{MORE} cold vulnerable in spring than less hardy varieties. \\ 

How will earlier budburst dates interact with potentially more common false springs or late winter frosts (tardive frosts)? See \cite{Sgubin2018} for more info on this. How regionally variable is this?\\

\section{Trait Plasticity} 
Does the same vine reach the same hardiness each year, conditional on temperature? Does the same variety or clone? How plastic is cold hardiness? \\

What about how variable varieties are between sites? If I take the same varieties, and look at how they differ relatively at different sites, will some varieties differ more? I.e. be more plastic? Or are acc/deacc rates unchanging?\\


\subsection{How does winter hardiness correlate with other important things?}
What would I expect it to correlate with? What is an important thing?
\begin{itemize}
\item heat and/or drought tolerance. Will new areas really become available because of warmer winter temperatures, or will these areas become to extreme during the summer?
\item Wine quality/verasion timing/phenology{}
\item yield
\item where does cold hardiness fit into syndromes?
\item what trade-offs are there physiologically in becoming and maintaining cold hardiness?  
\end{itemize}

\cite{Ferguson2014} – more variation in Hc max than Hc initial between varieties. Doesn't that suggest that some varieties must get hardier quicker? How does that work physiologically, and what are the costs of this? Does microclimate affect the rate of change of hardiness in relation to temperature? Or will Riesling always get (for example) 1 degree more cold tolerant for a unit of chilling whereas Merlot only 0.5 degrees of cold tolerance? If rate does change, would I expect more variation for a hardier grape? If the rate is invariable I would expect less between site variation (if micro climates are taken into consideration). \\

\section{Hardiness modeling}

What about growth modeling? I would need to make sure temperatures were all positive though. \\

Another dynamic model that is much simpler than the \cite{Ferguson2011} model, was made for Cherries by \cite{Salazar-Gutierrez2020} used a quadratic equation to model the observed temperature changes over the winter (I think?). With these temperatures they then built a non-linear model that regressed maximum temp, teh last two day's hardiness values and and the cumulative Julian day (squared) against today's hardiness. This model dealt with different cultivators separately. I like this model because it is relatively simple and allows plants to gain and lose hardiness. I should try it, but where would I allow variety and site variation? A less good thing about this model is that it is not just regressing against a temperature - you need to look at the temp of each day in order. 
\begin{equation*}
Y_{jk} = B_{0} + B_{1}Y_{j-1k} + B_{2}Y_{j-2k} + B_{3}d^{2}_{jk} + B_{4}mxt_{jk} + \epsilon_{jk} 
\end{equation*} 
In this equation $Y_{jk}$ is the modeled LTE value for percentage j and day $k$. So when they model LTE50 j is 50. Parameter $B_{0}$ is the intercept and parameters $B_{2}$, $B_{3}$ and $B_{4}$ are slopes in the model. Parameters $B_{1}Y_{j-1k}$ and $B_{2}Y_{j-2k}$ are yesterday's and the day before's hardiness value. Parameter $d$ is the number of days since the start of season where cherries acquire/have hardiness. Note it is squared!. Parameter $mxt_{jk}$ is the maximum temperature on day $k$. \\

Processed based modeling – can I think of a simpler one that would still work? \\

Should I focus on maximum hardiness or the rate of deac/acc? (related to potential risk of False Springs vs very cold periods midwinter). \cite{Charrier2013} mentioned that it has been observed that maximum hardiness achieved in winter is not dependent on environment (need to take another look at this).\\

The \cite{Ferguson2011} model was not good at predicting late winter/spring hardiness. It tended to overestimate hardiness a lot because the model didn't understand the physiological changes taking place as vines get ready for budburst. The \cite{Ferguson2014} model estimated hardiness based on phenological stage for the spring temperatures, and that seemed to work better. They said that LTE values derived from the lab were not correct for buds that were not dormant – something to do with water chemistry I think?\\

Maybe have a model that has two different rate periods, one before spring  bud physiological changes move to budburst, and one after this point. But how would I estimate this pivot date? \\

Maybe I could include a logistic component to my regression (like \cite{Ferguson2011}) to stop hardiness increasing linearly when it’s really cold? \\

\cite{Lenz2016} - in trees, budburst happened a certain number of days after the last time temperatures went below the freezing tolerance of new leaves. \\

Could I just include a model estimating chilling requirement and budburst (i.e. \cite{Caffarra2010}) and include that as the dates when I expect rates of change of hardiness to change? \\

\cite{Caffarra2010} – model Chardonnay phrenology using three sub-models (budburst, flowering and verasion). The model suggests phenotypic plasticity because mountain vines didn't react the same as all the non-mountain vines. \\  

\cite{Cortazar-Atauri2009} modeled budburst for varieties in France, and found models considering dormancy outperformed those not considering dormancy. \\  

Maybe I should have two slopes year year, one for acclimation period and one for deacclimation period? I would need to choose the switch from end to ectodormancy though, I guess. And still need to account for non-linear relationship at extreme values. \\ 

Could I just use a break point model to get a slope for autumn and a slope for spring? Bet breakpoint to midwinter or to change to ectodormancy? \\

Maybe I could have a transformation so that the relationship is not linear? Vines achieve winter hardiness using supercooling of intracellular water (see \cite{Kovaleski2018a} intro), and the maximum hardiness possible with this mechanism is 40\textdegree C (Biggs, 1953 in Kovaleski et al 2018 ). Maybe I could use this number in the non linear transformation? \\

Maybe uses a logistic regression? \cite{Kovaleski2018a} used a logistic regression to get deacclimation rate. Can it be used for non binary data?\\

\cite{Kovaleski2018a} found that different varieties of winegrape had different rates of deacclimation. \\ 

Deacclimation happens much faster than acclimation in plants \citep{Kalberer2006}. So I expect two different curves for change in hardiness with air temp during endo and ectodormancy. \\ 

How can I get at deacclimation rates through modeling? And how would I predict deacclimation rates should correlate with general hardiness?\\

It has not been done yet that I know of, but I could try using growth models. This would be cool if it works because it would hopefully take into consideration that hardiness doesn't change as much once it gets really cold.\\ 

\subsection{Physiology and hardiness mechanisms}  
Energy is necessary to drive acclimation and sugars play an essential role in freezing tolerance, the
resiliency of photosynthesis under stress conditions and how photosynthates are utilized (growth vs acclimation) also need to be better understood (\citep{Gusta2013}. Could there be a growth cost to being more cold hardy? Do more cold hardy vines produce smaller yields? \\

Proteins work in concert with sugars to establish cold tolerance. If plants haven't got enough sugars (carbohydrates) from the growing season then they cant be cold hardy \citep{Gusta2013}. \\

There is a model based on carbohydrate amounts in cells that models cold hardiness in Walnut trees. \\ 

What are the physiological differences between a more and a less cold hardy vine variety? \\

Lots of references in \cite{Lenz2016} about how budbreak is closely related to loss of winter hardiness. Expect more cold tolerant leaves to mean earlier budbreak because the risk of frost is lower. Is that why ,ore cold hardy vines budbreak earlier? Or do they budbreak earlier for other reasons? Are the leaves of more generally cold hardy varieties more cold hardy too? \\
  
\cite{Lenz2016} – suggest freezing resistance is rather fixed trait in trees. Freezing resistance was very similar among distinct populations of the same species. SO its easier to adjust leaf-out date than how cold hardy your leaves are. \\

Phenological response to temperature seems to be quite plastic based on elevation. For example budburst should change based on climate conditions (see \citep{Caffarra2010}. So should hardiness be plastic also? - Yes. But should the rate of acc/deacc also be plastic? Maybe this relates to the physiological mechanisms underpinning winter hardiness? \\

i\\
o\\
\\

\section{Model structure}

\begin{equation*}
ltePred_{i} \sim Normal(\mu_{i}, \sigma ) 
\end{equation*}
\begin{equation*}
\mu_{i} = \alpha_{var,i} + \alpha_{site, i}+ \beta_{var, i}\ast x_{i}
\end{equation*}

\begin{equation*}
\begin{bmatrix}
\alpha_{var} \\
\beta_{var}
\end{bmatrix}
\sim MVnorm
\left(
\begin{bmatrix}
\alpha \\
\beta
\end{bmatrix}
,S
\right)
\end{equation*}

\begin{equation*}
S = 
\begin{pmatrix}
\sigma_{\alpha Var} & 0 \\
0 & \sigma_{\beta Var} 
\end{pmatrix}
\rho
\begin{pmatrix}
\sigma_{\alpha Var} & 0 \\
0 & \sigma_{\beta Var} 
\end{pmatrix}
\end{equation*}

\begin{equation*}
\alpha_{site} \sim normal(0, \sigma_{\alpha Site}) 
\end{equation*}

Priors:
\begin{gather*}
\beta \sim lognormal(0,1)\\
\sigma \sim truncNormal(0,5)\\
\sigma_{\alpha Var} \sim truncNormal(0,5)\\
\sigma_{\alpha Site} \sim truncNormal(0,5)\\
\sigma_{\beta Var} \sim truncNormal(0,1)\\
\rho \sim LKJcorr (2)
\end{gather*}


\subsection{Quadratic Model Structure}

\begin{equation*}
ltePred_{i} \sim Normal(\mu_{i}, \sigma ) 
\end{equation*}
\begin{equation*}
\mu_{i} = \alpha_{var,i} + \alpha_{site, i}+ \beta_{var, i}\ast x_{i} + \beta_{quad} \ast x^{2}_{i}
\end{equation*}

\begin{equation*}
\begin{bmatrix}
\alpha_{var} \\
\beta_{var}
\end{bmatrix}
\sim MVnorm
\left(
\begin{bmatrix}
\alpha \\
\beta
\end{bmatrix}
,S
\right)
\end{equation*}

\begin{equation*}
S = 
\begin{pmatrix}
\sigma_{\alpha Var} & 0 \\
0 & \sigma_{\beta Var} 
\end{pmatrix}
\rho
\begin{pmatrix}
\sigma_{\alpha Var} & 0 \\
0 & \sigma_{\beta Var} 
\end{pmatrix}
\end{equation*}

\begin{equation*}
\alpha_{site} \sim normal(0, \sigma_{\alpha Site}) 
\end{equation*}

Priors:
\begin{gather*}
\alpha \sim Normal(-15, 12)\\
\beta \sim lognormal(0,1)\\
\beta_{quad} \sim Normal(0,1)\\
\sigma \sim truncNormal(0,5)\\
\sigma_{\alpha Var} \sim truncNormal(0,5)\\
\sigma_{\alpha Site} \sim truncNormal(0,5)\\
\sigma_{\beta Var} \sim truncNormal(0,1)\\
\rho \sim LKJcorr (2)
\end{gather*}


\bibliographystyle{/home/faith/Documents/Bibtex/styles/besjournals/besjournals}
\bibliography{/home/faith/Documents/Bibtex/mendeleyStuff/library}

\end{document}



