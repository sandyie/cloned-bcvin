\documentclass[11pt,letter]{article}
\usepackage[top=1.00in, bottom=1.0in, left=1.1in, right=1.1in]{geometry}
\renewcommand{\baselinestretch}{1.1}
\usepackage{graphicx}
\usepackage{natbib}
\usepackage{amsmath}

\def\labelitemi{--}
\parindent=0pt

\begin{document}
\bibliographystyle{/Users/Lizzie/Documents/EndnoteRelated/Bibtex/styles/besjournals}
\renewcommand{\refname}{\CHead{}}

\title{Safety plan for Wolkovich Lab: Okanagan field work}
\date{\today}
% \author{}
\maketitle
\tableofcontents

\section{Personnel, Contact Information and Safety Training}
\section{Identifying and Addressing Specific Health and Safety Concerns}
\subsection{COVID-19}
\subsection{Location}
\subsection{Interactions with collaborators}
\subsection{Interactions with field workers in vineyards}


\end{document}

The fieldwork in the Okanagan should be OK as long as your crew is small, there are no more than two people in a vehicle together, one driving and one in the back right seat (unless from the same household), social distancing can be maintained while working and in accommodations (unless from the same household), and there is no contact with indigenous communities. I've attached a few exemptions that have been approved, and here's a useful link to the WorkSafe BC recommended safety practices for forestry fieldwork that has largely been adopted by some of the other applications (Simard and Hinch). https://www.worksafebc.com/en/about-us/covid-19-updates/covid-19-industry-information/forestry I know working in a vineyard and in a remote field site are different, but there are still transportation, social distancing, PPE and sanitation concerns. You will also need to address how you will interact with or avoid farm workers at these sites. I would also say that you should be able to get the plots set up, but should defer any data collection that can wait for next year.