\documentclass[11pt,letter]{article}
\usepackage[top=1.00in, bottom=1.0in, left=1.1in, right=1.1in]{geometry}
\renewcommand{\baselinestretch}{1.1}
\usepackage{graphicx}
\usepackage{natbib}
\usepackage{amsmath}

\def\labelitemi{--}
\parindent=0pt

\begin{document}
\bibliographystyle{/Users/Lizzie/Documents/EndnoteRelated/Bibtex/styles/besjournals}
\renewcommand{\refname}{\CHead{}}

\title{Wolkovich Lab \\ Research exemption request for critical research}
\date{\today}
% \author{}
\maketitle
\tableofcontents


\section{Overview of work}
\section{Reducing field work to critical aspects}
\section{Addressing UBC's specific questions}
Here we respond directly to UBC questions regarding social distancing, conduct the research, and follow current Government and University travel advisories and restrictions.\\

\emph{1. Are you able to travel to and from your research sites in compliance with current University and Government Travel Restrictions?}\\

\emph{2. How does the size of your team allow for social distancing?}\\

\emph{3. How do modes of travel at your research site allow for social distancing?}\\

\emph{4. How do living and working conditions at the site allow for social distancing and/or for self-isolation should it become necessary?}\\

\emph{5. What are the plans in place should a member of the research team develop COVID-19 symptoms?}\\

\emph{6. If your research will bring you into contact with local communities, how will you ensure required and effective social distancing protocols?}\\


\end{document}