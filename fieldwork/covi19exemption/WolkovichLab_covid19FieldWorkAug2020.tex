\documentclass[11pt,letter]{article}
\usepackage[top=1.00in, bottom=1.0in, left=1.1in, right=1.1in]{geometry}
\renewcommand{\baselinestretch}{1.1}
\usepackage{graphicx}
\usepackage{natbib}
\usepackage{amsmath}
\usepackage{hyperref}
\usepackage{textcomp}

\def\labelitemi{--}
\parindent=0pt



\begin{document}
\bibliographystyle{/Users/Lizzie/Documents/EndnoteRelated/Bibtex/styles/besjournals}
\renewcommand{\refname}{\CHead{}}

\title{Wolkovich Lab \\ Research exemption request for critical research}
\date{\today}
% \author{}
\maketitle
% \tableofcontents

Thank you for considering our request to conduct brief but critical field work for our research program in the Okanagan Valley focused on climate change impacts and adaptation. The immediate work we are requesting an exemption for is part of a four-year collaborative project. This document addresses the questions outlined by the UBC, conforms with current guidelines from WorkSafe BC guidelines both for Agriculture Field work and for Forestry Field Work and COVID-19 safety and follows guidelines for Self-Isolation and Symptom Tracking for COVID-19 from the BC Centre for Disease Control.\\

We additionally have already (and separately) provide a full UBC Safety Plan for this work. The safety plan is required by WorkSafe BC and has been filed with the Department of Forest and Conservation Sciences. \\

This document is an update to our exemption requested and approved in April 2020. 

\section{Overview of work}
The field crew will be based in Penticton (49.481024, -119.587796). Fieldwork will involve driving to various vineyards within a 30km radius throughout the Okanagan valley. At each site, crew members will work out logistics for collecting berry samples across our sample vines, blocks and vineyards. Once this protocol is established, it will be followed by a local contractor. \\

The proposed field work is critical to a four-year project, funded under the BC FAIP Climate Change Adaptation Program (`Modelling winegrape phenology for a warming Okanagan'), and represents the first project relating to Canadian winegrowing for PI Wolkovich, who recently started at UBC (January 2018). Thus failure to successfully establish this work will jeopardize the four-year project and may significantly jeopardize the lab's developing research program. 

\section{Reducing field work to critical aspects}
Our initial plan was to collect three year's of phenological (vine timing i.e. flowering) monitoring data in the Okanagan Valley vineyards so we can build phenology models to help local vineyards and vineyards globally adapt to climate change. It is essential for our models that we obtain more than one year of phenology data because phenology is variable between years and is very responsive to climactic differences. Given current conditions, we recognize that we cannot collect a season of phenology data this year (which would take 4-5 months of field work).\\ 

Instead we have significantly pared down our 4-5 month planned field program to a proposed plan $<$ 1 month work to establish field sites, and standardize protocols with our collaborators this year so that we can successfully collect data in the next two years. We here request up to a 5 day visit in late August to finalize the last field protocol (berry collection). We have additionally scaled back the size of the field crew, from 5-6 field researchers (including undergraduates) to three senior researchers (PI Wolkovich, one postdoctoral researcher, Legault, and one MSc student, Garner).\\

Establishment of field work this year is critical to any future field work because our key contacts with the vineyards and for our sampling design, Dr. Pat Bowen and Carl Bogdanoff, MSc (both from Agriculture Canada, Summerland \href{https://profils-profiles.science.gc.ca/en/research-centre/summerland-research-and-development-centre}{link to centre here}, are retiring this year and cannot assist us later this spring/summer or next year. Without these connections we do not have the support to appropriately establish sampling plots as outlined in our proposal. Bowen and Bogdanoff critically provide knowledge of climatic diversity, management regimes and their previous sampling locations to allow us to establish robust plots for our project. This year's establishment of field sites is therefore highly influential for the long-term research output and industry collaboration with the Wolkovich lab at UBC. It is also crucial that we visit the vineyards in early spring rather than any other time, because this is the window of time where the vineyards are most comfortable with allowing us access to their sites. Earlier in the year snow cover can cause problems. Later in the year much of the viticultural practices take place, so the vineyard staff must focus on essential maintenance rather than incoming researchers. This year we will thus focus on getting sites set up, and in doing so, lay the foundations for a successful collaboration with the Okanagan vineyards. Unless there are substantial changes to UBC field work policies, data collection will take place next year onwards.

\section{Addressing UBC's specific questions}
We are following the principles and guidelines laid out by WorkSafe BC specifically for Forestry field work and COVID-19 safety \url{https://www.worksafebc.com/en/about-us/covid-19-updates/covid-19-industry-information/forestry} and additionally for Agriculture field work \url{https://www.worksafebc.com/en/about-us/covid-19-updates/covid-19-industry-information/agriculture}.\\

\emph{1. Are you able to travel to and from your research sites in compliance with current University and Government Travel Restrictions?}\\
Yes, transit to field sites will take place in two separate vehicles (details below in response to question 3) and only healthy team members will be allowed to travel. Any crew member that has had a confirmed case of COVID-19 or who is unwell at the time of departure based, will not be allowed to take part in this research project. During field work and for two weeks after crew members will monitor closely for symptoms and maintain a record of daily personal temperatures. Temperature screening will be done with a properly calibrated thermometer, with temperature thresholds based on Health Canada guidelines. An indicator for possible COVID-19 infection is 37.5 $^{\circ}$C. Active daily monitoring will be conducted using the Personal Health Daily Monitoring Form (from the BC Center for Disease Control). \\
   
\emph{2. How does the size of your team allow for social distancing?}\\
The crew consists of three active field members plus an additional driver who will not be in the field. This additional driver is Lizzie's partner Jonathan, who will be available to share driving with Lizzie but will not undertake active field duties in vineyards. Other than driving to and from the accommodation, and driving to the field sites, all of the work is outdoors, and there will be no difficulty in maintaining appropriate social distancing of two meter minimum distances. Any team meetings will take place outside to maintain physical distance. All team members will be responsible for ensuring physical distancing is maintained and any special COVID-19 precautions are strictly followed as per WorkSafe BC guidelines. Good hygiene will be practiced at all times, in line with the WorkSafe BC recommendations. \\

\emph{3. How do modes of travel at your research site allow for social distancing?}\\
Lizzie will be in her own personal vehicle along with her partner Jonathan, while Legault and Garner will share a field vehicle. Lizzie and her partner Jonathan Davies are already isolating as a family unit so they do not need to maintain social isolation from each other. In the vehicle shared by Legault and Garner, the passenger will sit in the back-passenger seat diagonally positioned relative to the driver to maintain a two meter distance ensuring physical distancing is in place as per WorkSafe BC guidelines (the field vehicle is a 8-seat SUV). All crew will wash hands with soap and water (or with hand sanitizer) prior to entering the vehicles. The interior (e.g. seatbelts, headrests, door handles, steering wheels, and hand holds) and exterior door handles of both vehicles will be wiped down with sanitizing cleaner at the end of each day, or sooner if a member of the crew reports COVID-19 symptoms.\\  

\emph{4. How do living and working conditions at the site allow for social distancing and/or for self-isolation should it become necessary?}\\
All team members will stay in a large house near Penticton. They will have separate bedrooms, and will sanitize communal space and equipment regularly and remain 2 meters apart at all times. Each crew member will have their own sanitizer and hygiene supplies for personal use for the duration of the field work. There will also be bulk liquid soap, water, and sanitizer that the crew can use to fill their personal containers if the need arises.\\

\emph{5. What are the plans in place should a member of the research team develop COVID-19 symptoms?}\\
If a crew member exhibits COVID-19 symptoms (e.g. sore throat, fever, sneezing or coughing), they would be put into the far back seat of their SUV, and instructed to put on a protective face mask and gloves. The other crew member, if not already the driver, would become the driver and would put on a protective face mask and gloves. The inside of the vehicle (e.g. seatbelts, headrests, door handles, steering wheels, and hand holds) would then be wiped down with sanitizing cleaner. The vehicle would immediately head back to Vancouver not stopping along the way (there will extra food/water in the SUV). The drive time would be up to but likely not more than 6 hours. During the drive, the person with symptoms would continue to monitor their situation using a thermometer and tracking other health metrics. The field crew have thermometers in the vehicles to enable effective monitoring. The vehicle interior and exterior door handles would be wiped down with sanitizing cleaner upon completion of the drive. Following the protocols set out by WorkSafe BC, both the person with symptoms and the driver would go into self-isolation in their homes for two weeks. The research supervisor (Wolkovich) would immediately notify the Public Health Authority in British Columbia by dialing 811 that someone with symptoms is self-isolating. We would notify the management of our field accommodation so that they could take necessary cleaning precautions. Any crew that has had a confirmed case of COVID-19 will not be allowed to come to work.\\

\emph{6. If your research will bring you into contact with local communities, how will you ensure required and effective social distancing protocols?}\\

The field crew will need some infrequent interaction with local collaborators, but will maintain social distance from then as per the WorkSafe BC recommendations and use non-contact whenever possible (e.g., asking collaborators to visit sites after they have been flagged to assess locations). To this end, all meetings will be kept to a minimum, and will take place outside where there is enough space for 2 meters between every person. The field crew will also practice good hygiene, including wearing face masks, not touching faces, washing hands often and for at least 20 seconds (or sanitizing hands if washing not possible) and sneezing/coughing into disposable tissues if needed. \\

To avoid contact with vineyard field workers, we have the following plan in place. The field crew will alert vineyards to our presence on site so the field crew and vineyard field workers will be aware of each other's presence. Our fieldwork does not require interaction with such workers, so if the field crew see any workers they will retreat to a safe distance until they have left the immediate area. Any surfaces the field crew and vineyard fieldworkers might come into contact with will be sanitized before and after contact. Our crew members will also maintain a 2 meter distance from each other at all times as per the WorkSafe BC guidelines. \\

The field base will be close to the town of Penticton, where the field crew will obtain necessary groceries and gas. To reduce risk to local communities, shopping trips will be kept to a minimum (one trip), during shopping researchers will wear face masks, and sanitize hands before and after shopping, and maintain good hygiene. The field crew will only use self-service gas stations and prior to use we will wipe down the gas pump nozzle and PIN pad on the pump with sanitizing wipes. The field crew will not come into contact with any First Nation communities in the Okanagan Valley.\\ 

\clearpage

\section{Contact information for field team}


\begin{table}[h!]
\caption{Field crew contact details. Please see the safety plan (filed with the Department) for emergency contact information; and additional Okanagan contact information. } % title of Table
%\centering % used for centering table. I dont want my tabel centred 
\begin{tabular}{ l | l | l }  % left jusitfy columns (2 columns)
\hline\hline %inserts double horizontal lines
Name & Contact\\ [0.5ex] % inserts table
%heading
\hline % inserts single horizontal line
Elizabeth Wolkovich & 603 667 5099 & \url{e.wolkovich@ubc.ca}  \\ % inserting body of the table
Geoff Legault & 778 952 2217 &\url{geoffrey.legault@ubc.ca}\\
Mira Garner & 608 228 5215 & \url{mira.garner@gmail.com}\\
Jonathan Davies & 604 822 5486 & \url{j.davies@ubc.ca} \\ 
\hline %inserts single line
\end{tabular}
\label{table:nonlin} % is used to refer this table in the text
\end{table}


\section{Addressing UBC's updated questions}
\textbf{As you prepare for the field, will you need to access your lab or office?}
Yes, access to the Wolkovich forestry lab (Forestry 3431) will be required to check gather field supplies and receive some items ordered. Garner, Legault and Wolkovich are approved for re-entry to the FSC for this purpose. \\

\textbf{Are you able to travel to and from research sites in compliance with current government and University travel advisories and restrictions?}
Yes, as outline above, see `Addressing UBC's specific questions' \#1 above.\\

\textbf{How will the field team composition and size allow for physical distancing?}
The field team will consist primarily of three people, which will allow for proper physical distancing at all times especially since work is outdoors. In addition, the field team members will practice physical distancing with others outside of their immediate household, practice good personal hygiene. If any individuals have any COVID symptoms, they will not be allowed to go on the research trip. \\

\textbf{Will you require permission from a government department, non-profit agency or research institute to conduct work at your field site?  If so, please attach signed permissions.} 
No. \\ 

\textbf{What are the proposed dates for this fieldwork?}
The proposed dates of field work is 25-29 August.\\

\textbf{Is there a contact number for your team during the fieldwork period?  If not, how can the team be reached daily while in the field?} Yes, please refer above for contact details.\\

\textbf{How will you train the field crew in COVID-19 safety practices?} 
Yes. Each team member will thoroughly review BC's recommended COVID-19 safety procedures \url{http://www.bccdc.ca/health-info/diseases-conditions/covid-19/common-questions} and WorkSafe BC's guidelines for Forestry and Agriculture field work and COVID-19 safety \url{https://www.worksafebc.com/en/about-us/
covid-19-updates/covid-19-industry-information/forestry}. \\

\textbf{How do modes of travel at your research site allow for physical distancing?}
See `Addressing UBC's specific questions' \#3 above. \\

\textbf{How do living and working conditions at the site allow for physical distancing and/or for self-isolation should it be necessary?}
See `Addressing UBC's specific questions' \#4  above. \\ 

\textbf{What plans are in place should a member of the research team develop COVID-19 symptoms?}
See `Addressing UBC's specific questions' \#5 above. \\ 

\textbf{If your research will potentially bring you into contact with local communities, how will you ensure required and effective physical distancing?}
Proper physical distancing (minimum 2 m distance apart) will take place with anybody in the local communities of the Okanagan especially around the field sites. \\

\textbf{When you return from the field, will you need to access your lab or office to process or handle samples?  If so, please describe the expected duration of sample-handling and why the handling must be done immediately.}
Yes, we will return with grape samples and need access to the lab (Forestry 3431). This access has already been approved for Phase 2 of the FSC plan.\\ 

\textbf{Please confirm that this Safety Plan has been shared with staff and students through email and will also be made available as a shared document.   Staff and students can either provide a signature or email confirmation that they have received, read and understood the contents of the plan.}\\

Yes. PDFs of emailed consent are available upon request.


\end{document}