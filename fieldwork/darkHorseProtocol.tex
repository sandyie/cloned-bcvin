\documentclass[11pt,letter]{article}
\usepackage[top=1.00in, bottom=1.0in, left=1.1in, right=1.1in]{geometry}
\renewcommand{\baselinestretch}{1.1}
\usepackage[
singlelinecheck=false % <-- important
]{caption}
\usepackage{hyperref}%helps with the email address 
\usepackage{graphicx} %including pictrures 

\def\labelitemi{--}
\parindent=0pt

\def\labelitemi{--}
\parindent=0pt

\newenvironment{smitemize}{
\begin{itemize}
  \setlength{\itemsep}{0pt}
  \setlength{\parskip}{0.8pt}
  \setlength{\parsep}{0pt}}
{\end{itemize}
}

\graphicspath{ {./protPhotos/} }% tell latex where to find photos 

\begin{document}
\bibliographystyle{/Users/Lizzie/Documents/EndnoteRelated/Bibtex/styles/besjournals}
\renewcommand{\refname}{\CHead{}}

\title{Wolkovich Phenological Monitoring Protocol\\
Dark Horse Site}
\date{ }
\maketitle
\tableofcontents

\section{Objectives}
We aim to characterize the phenology of a large number of V. vinifera varieties planted in vineyards around the Okenagan Valley area in BC. Our primary goal is to improve grower models of phenological timing with relation to climate, To do this we need to understand how genetics and correlations with other plant traits fundamentally drive the hyper-phenological diversity of V. vinifera. To this end, we aim to sample a large diversity of phenologies, from very early to very late varieties. BDue to the difficulties of undertaking fieldwork in a time of Corvid-19 spread, we were not able to undertake all the surveying we wanted this year, but Mike Watson has agreed to some surveying through Arterra employees at the Dark Horse vineyard.

\section{Sampling population at Dark Horse}
(describe how many vines and where)

\section{Sampling frequency and timing}
\begin{enumerate}
  \item Lizzie's team (Mira and Faith) flagged vines and started surveying phenology in early May. We did not get out early enough for budburst. 
  \item We will aimed to monitor four phenological stages, but missed budburst:
  \begin{enumerate}
	\item Budburst (approximately mid April/early May). Until EL stage 9, record the EL stage of buds on three spurs. Once EL stage is at 9 you can stop recording until flowering starts (note that you may need to record higher than stage 9 at times in order to record whatever stage you see after 8, even if it is 12 or such if you have not yet recorded stage 9, more on this below).
	\item Bloom (approximately May/early June)
  	\item Verasion (approximately mid July-mid August)
  	\item Ripening (Brix - approximately mid August-end September)
  \end{enumerate}
  \item Our intention is ho bring someone back at least for Brix sampling, so we mostly need help with flowering and verasion monitoring. 
  \item We aim for sampling at least once a week. If you have more time, especially if it has been a warm week, then twice a week would be even better.

\end{enumerate}

\section{Finding the right buds to monitor}
We have flagged three buds/shoots on each of the vines in our monitoring program. We randomly chose one of the canes or cordons on each plant, and flagged three arms or spurs on the chosen cordon (Figure \ref{fig:CordonPruned}) or three buds/shoots on the chosen cane (Figure \ref{fig:CanePruned}). We flagged as close to the bud as we could when we initially placed the flagging tape, and later flagged some of the shoots once they were big enough. For the cordon pruned vines, we monitor the lowest bud/shoot of the highest spur of the arm (Figure \ref{fig:TwoSpurs}). 

\section{Establishing clusters}
\begin{enumerate}
	\item Once the shoot we are monitoring becomes well developed (E-L stage 17 or later) you can move the flagging to a specific bloom. This is probably best done as you move to monitoring \% bloom. 
	\item We will sample one cluster per shoot. 
	(I dont know how they should chose the clusters)

\end{enumerate}

\section{Observing Phenology}
We are using the Eichhorn-Lorennz system for monitoring phenology described in this PDF: \\

Our goal is to have standardized observations of phenological stages for analysis. There are four main widely observed stages for grapevine: (1) budburst, (2) flowering or bloom, (3) veraison, and (4) ripening and harvest. For the first stage, budburst, we will use the stage numbers from the Modified Eichorn Lorenz system (a stage number between 1-17, with 4 meaning budburst; see Figure \ref{fig:ELScale} below).For stages 2 and 3, we will estimate the percent occurrence of the stage (proportion of berries on a selected cluster have gone through the stage of bloom or color change, respectively). We will measure ripening quantitatively, using a refractometer to measure Brix (sugar) accumulation. 

For each vine:
\begin{enumerate}
	\item Stand facing the vine. Double-check Plant ID (labeled with a sharpie on one of the flags), block row number and bud number before recording in correct place on data sheet that we will provide. We don't keep information on the varieties within each block on the field data recording sheets.
	\item For pre-bloom, record the appropriate E-L stage of the shoot that was flagged, or the shoot closeted to the flagging that matches our selection criteria laid out in the section above. Record its EL stage (number from “1”: still dormant, to “17”: twelve leaves separated- see \ref{fig:ELScale}).
	\item If the monitored shoot is removed reflag a lower shoot shoot and monitor that one from now on. Record this change on the data sheet. 
	\item  For bloom (EL 23) and veraison (EL 35), look at each cluster (\#1, \#2, \#3) and estimate the percent (from 0-100\%) of berries on the sample cluster that have achieved that stage. 
	\item Take photos of representative illustrations for different varieties of target \% at stage (e.g., clusters at 5\%, 25\%, 50\%, 75\%, 95\% bloom) for different varieties. aim for 3-5 photos of each percentage taken across a diversity of varieties and across a couple different sampling dates, and make sure vine number and date are visible in photo for identification.
	\item For ripening, our current plan is to send one of our team to collect berries so we can analyze Brix. We will aim to collect samples starting around 18° Brix until commercial harvest level. 

\end{enumerate}



\section{Establishing clusters}

\section{What to do with the data}

\section{Contact details}



\end{document}

This is a short protocol designed for Mike and his team to continue gathering data while we are away. 
https://docs.google.com/document/d/13JRKCdJJOEcVo9Vh5Hub2SRHbJnhbqi8qDhj7a_kgiA/edit#