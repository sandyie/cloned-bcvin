\documentclass[11pt,letter]{article}
\usepackage[top=1.00in, bottom=1.0in, left=1.1in, right=1.1in]{geometry}
\renewcommand{\baselinestretch}{1.1}
\usepackage{graphicx}
\usepackage{natbib}
\usepackage{amsmath}
\usepackage{textcomp}
\usepackage[
singlelinecheck=false % <-- important
]{caption}
\usepackage{float}
\usepackage{hyperref}

\def\labelitemi{--}
\parindent=0pt

\def\labelitemi{--}
\parindent=0pt

\newenvironment{smitemize}{
\begin{itemize}
  \setlength{\itemsep}{0pt}
  \setlength{\parskip}{0.8pt}
  \setlength{\parsep}{0pt}}
{\end{itemize}
}


\begin{document}
\bibliographystyle{/Users/Lizzie/Documents/EndnoteRelated/Bibtex/styles/besjournals}
\renewcommand{\refname}{\CHead{}}

\title{bcvin Fieldwork Log, May 2020}
\author{}
\maketitle
\tableofcontents

\subsection{4 May 2020}
\begin{smitemize}
\item Met with Pat Bowen and Carl Bogdanoff to discuss sampling and vineyard etiquette.
\item We should be allowed to leave up tape throughout the season and mark buds with metal tags but should ask first, of course
\item May be allowed to tag vines until next year for easy finding - ask managers
\item Pat and Carl don't think Arterra or Quail's Gate are doing organics management. Sebastian Farms is all organic now.
\item Merlot or Chardonnay would be good varieties to sample more throughly because they are commonly grown and popular for the region.
\item Slope aspect, east vs west side of lake (getting morning or afternoon sun), row direction - interests for climatic variation
\item the vineyard managers we are meeting should be considered more collaborators than facilitators. They might have good ideas.
\item to keep vine quality consistent, we should focus on vineyards directly managed by the big wine companies rather than contract growers. This is because contract growers get paid based on the amount of grapes they sell, rather than the quality of the grapes, so there is a tendency to overcrop. Mike may suggest a contract grower though that is growing to the higher standards, and if so we should go with his suggestion. As long as we keep to high quality vine management, the wine company should not make a big difference.
\item the most popular grape varieties shifts regularly. At the moment there is a lot of shifting to merlot from white winegrapes in the south. How it is probably 70:30 ratio of red to white there. More centrally in the valley it is more like 50:50. Sav blanc is getting more popular, and pinot gris is also popular at the moment
\item vines should be at least 4 years old for us to monitor them

\end{smitemize}

Grower communication
\begin{smitemize}
\item send growers information in easily digestible and usable formats i.e. simple spreadsheets or maps
\item make yourself useful by giving growers information that they want
\item Try never to say no to a talk or committee meeting invitation
\item Get on the Research and Development Council for BC Wine. Maybe attend conference if there is one.
\item when writing grants, deliverables must be useful.

\end{smitemize}

Need to consider: Make sure vines we choose will be there for 3 years
\vspace{1.5ex}

 We also briefly discussed winter hardiness modeling.
 \begin{smitemize}
\item Carl has started using a 4th order Quadratic equation, and this seems to be working well
\item acclimation and deacclimation are not the same process, but Carl thinks they should have similar rates
\item Carl also doesn't think there should be a relationship between maximum hardiness and rate of deacclimation. He pointed out Riesling bursts quite late, whereas Chardonnay bursts early, but they are both fairly hardy

\end{smitemize}

\subsection{5 May 2020}
\begin{smitemize}
\item met with Mike Watson at Dark Horse Vineyard, vineyard manager?/head viticulturist? at Arterra to discuss site access and what to consider when choosing vines
\item Mike is interested in soil type. Soil type can differ a within blocks, although less so for Arterra because Mike tries to avoid this issue. For contract growers, especially new growers, it is more of a problem because they focus on maximizing yield per hectare, so squish in as many vines as possible
\item Mike also mentioned something about a contract grower who "pushed the vines too much", who 3rd leaf on one year and second leaf the next, but mostly got away with it, although there were "circles of death" at the site. Mike seemed to think this might be related to the soil. He said some growers can get away with doing everything wrong, while others cant, and maybe this is because of soil. We didn't understand this section of the conversation.
\item some site maps might be incorrect, so Mike will send us updated maps
\item Fine to sample at Dark Horse, Whitetail, McIntrye, NK'MIP Cellars.
\item Dark Horse - We need to let Mike know we will be around. If we keep things non-onerous they can collect some data for us.
\item No expected variety/plant changes in the any of vineyards.
\item Flagging is fine, and we don't need signs as we don't want the workers to do anything special. We should use flagging tape that is not orange, yellow or pink though. (Faith also saw some blue while out at the vineyards).
\item Be aware that Inkameep Winery and NK'MIP Cellars are different. We are working at NK'MIP Cellars.
\item Phone numbers (to be contacted the day before entering the vineyard)
\item Manjit Deol (Manager at McIntrye and Whitetail) - 250 488 8215
\item Scott Carlson (Vineyard supervisor at McIntrye and WHitatail) - 250 485 7920. Call preferentially to Manjit because on site more
\item NK'MIP - Nelson Dutra (not sure of surname spelling) - 250 485 8085
\item there are also boards up at each vineyard with the spraying schedule. If in doubt, call Mike.
\item Call the day before to let vineyard know we will be sampling.
\item NK'MIP is usually locked, but we have a key. We should return the key at the end of the season.
\item we should access the vineyards between 6am and 3/4 pm. That is when the workers are there.
\item weekends should be ok. The workers work Saturdays but not Sundays though, so maybe we should avoid Sundays?
\item Nothing is alarmed
\item Mike talked very enthusiastically about a GIS dashboard they are working on that will have spraying schedules and stuff. Faith wonders if we could make our information (when we have it) a GIS dashboard, or a layer for their dashboard?
\item when we asked Mike what phenological stages he looks for, he mentioned "when the inflorescence separates". He uses this to gage what pesticides to use (whether to combine sulfur and pertritas or not). He's not sure how accurate this stage is, and we are not totally sure which Eichhorn - Lorenz stage it is, but we should try and capture it. It's a bit later than we would have thought to measure.

\end{smitemize}

Other notes from our Adventures:
\begin{smitemize}
\item why do Arterra cane prune rather than cordon prune some of the vines? Also we should consider how our flagging will work with these pruning types. Probably we need to use loose flagging so we don't restrict cane growth
\item Why do they spray irrigate in Whitetail? Also avoid rows where there are irrigation issues.
\item Mike has friends who rent Apex condos during summer if we can't find housing
\item Shiraz is Syrah! New revelation for Mira!
\item NK'MIP Cellars and McIntrye have Shiraz/Syrah
\item All 4 Arterra vineyards have Cabernet Sauvignon - consider adding to variety list

\end{smitemize}

Dark Horse Vineyard Walk notes
\begin{smitemize}
\item We should have 6 vines next to each other as a unit, and always skip to the first pole to avoid vines right on the outside edge of the block
\item We want probably a ratio of 40:40:20 of upper:lower:middle vines in the row. We also want to be as efficient with our time as possible, so avoid unnecessary crossing vine rows.
\item Sample near weather station
\item Merlot and Chardonnay are across the road from each other so make sampling plan easy to grab both.
\item Sample Merlot from several blocks and Chardonnay from both blocks

\end{smitemize}

NK'MIP Vineyard Walk notes
\begin{smitemize}
\item Sample both Merlot blocks, both Shiraz blocks, Cab Sau?
\item 6 top, 6 bottom, a few in middle

\end{smitemize}


\subsection{6 May 2020}

Chad meeting at Quail's Gate Wineshop to discuss Quail's Gate Estate Vineyard and Mannhardt Vineyard
\begin{smitemize}
\item Syrah/shiraz at Quail's Gate is not doing very well. 
\item In fact, very few of the vines below Boucherie Rd are doing very well. They are likely to be removed. Chad thinks this might be because the soil down there is more clay, whereas the top is more volcanic soil and gravel 
\item The Sauvignon blanc is ok though, probably there for 5 years.
\item Chad was not sure if there were good soil maps (we asked Pat and Carl and they said they have maps of soil)
\item Chad said we should sample in the other Quail's Gate vineyard (Mannhardt), where they have Riesling. 
\item Flagging tape is fine, and we don't need signs. We should label the flagging tape though with UBC and maybe Lizzie's name?
\item There are tags on every second row with vine rootstock, variety clone and block info. 
\item Chad doesn't think different blocks will be very different phenologically. 
\item There are weather stations on site
\item Chad was interested in the connection between numbers of flowers and numbers of fruit produced. Maybe we could capture this using software?
\item In passing, Chad mentioned there had been 90 per cent die off of buds. Not sure why, but I think from cold damage?
\item we could get pruning dates from Chad. He would be interested to see how pruning dates affect phenology. He talked about a trial where they tried to put back the phenology with late pruning. They did delay budbreak but then veraison was earlier. 
\item clone vs cordon pruning. Chad prefers cane pruning for high quality grapes because there is less chance of incorrect pruning leaving too many buds. He would be interested if pruning style affects phenology. The Riesling in Mannhardt would be good to try and tackle this question because there is a mix of pruning within the same block. But its possible vines might transition during our study.
\item Chad is trying to shift pruning so that a block is entirely cane or entirely cordon pruned. Some blocks are still mixed so pruning of some plants may change during experiment unless we flag them.
\item Rootstock 3309 is preferred because middle yield and good scion merging, 101/14 is good for lower vigour, and 504 is good for higher vigour. 
\end{smitemize}

Sampling
\begin{smitemize}
\item Chardonnay 4-1 is going to be changed, so don't sample. 3-4b is better. Don't sample 3-4a because it has a disease. Maybe utopia or something?
\item Blocks 5-1 to 5-4 are fine. 
\item management is fairly consistent in the vineyard. Apparently there are a few differences though. For example some of the Pinot noir clones are better than others so are either managed for quality or quantity, 4-11 and 3-1 are high quality. Other blocks not so high quality. For Chardonnay the top blocks 5-1 to 5-3 are less high quality and lower blocks 5-2 to 5-4. This is due to a spring in the top plus soil differences. Also perhaps a different clone?
\item But there are some trials of heat treatment instead of pertitiside. Tanya is doing this and may have phenology data. It would be interesting to see if differences based on this, but the current trials are on vines that will not stay. Chad would like to try on Pinot noir, and if so maybe we could get data from the different trials. 
 \end{smitemize}  

Access
\begin{smitemize}
\item We should text Chad, or even better Judy the Assistant Manager
\item No preferences on timing access or weekend vs weekday visits. 6 am is fine. Workers finish about 3.30pm.
\item There is a carpark opposite block 3-4a we can park in (just before the main shop, on the other side, if you are driving north from Penticton)
\item walking access is fine. Car access seems fine in Mannhardt but not Quail's Gate main vineyard. 
\end{smitemize}

\subsubsection {Meeting with Carl, Lizzie and Faith 2.30 pm}
\begin{smitemize}
\item generally we talked about winter hardiness (see CarlMeeting2020 document in hardiness folder)
\item Carl also supported what Chad said about popular rootstocks
\item we should be careful about s=rattle snakes 
\item we should get an exiting GIS Dashboard 
\item Carl was less convinced about the benefits of cane pruning, but talked about the difficulties of getting good pruning staff.
\end{smitemize}

\subsubsection{Meeting with Lizzie, Mira, Faith - sampling planning}
\begin{smitemize}
\item 300 plants in a variety rich plot at Davis took 4-6 hours to sample when interns got used to sampling. There was little walking between plants.
\item Save 20\% time to add Sebastian Farms sampling next year
\item Aim to sample 6 plants next to each other, do not go below 4 plants
\item Drop plants per block or diversity of locations within block before dropping diversity of blocks and varieties. No less than 16 plants per block.
\item Keep multiple blocks of a variety if they have the same rootstock and clone. If there are differences, it's less important to keep multiple blocks.
\item If desperate to lower numbers, look for variety overlap between McIntyre and Whitetail. 
\item If needed, order of dropping varieties = 1. Riesling, 2. Cabernet sauvignon, 3. Syrah. Not dropping: Sauvignon blanc, Pinot noir, Chardonnay, or Merlot.
\item Pick same rootstock if possible to minimize rootstock effect (S04?)
\item Plan for Dark Horse sampling to take about 1-1.5 hours so we can ask them to continue sampling after we leave.
\item Definitely try to do Brix sampling so we can compare with Mike's Brix
\item Try to prepare some report for Dark Horse as a thank you for helping us sample.

\end{smitemize}

\subsection{7 May 2020}
\begin{smitemize}
\item Flagged in Whitetail: 36 plants in block J Riesling, 36 plants in Chardonnay block OIB, 36 plants in Riesling block B, 24 plants in Chardonnay block C, 36 plants in Sauvignon blanc block D, 36 plants in Cabernet sauvignon block G.
\item Flagged in McIntyre: 24 in Cabernet sauvignon block Q, 36 in Pinot noir block O.
\item Merlot was not tagged in Whitetail because they were in the process of removing some vines and planting new ones. Will check for update next time we go to Whitetail.
\item Flags in Whitetail OIB likely to be removed
\item Scott Carlson gave us a key for McIntyre (and maybe Whitetail gates) so we can stay late if needed.
\item Initially used old orange flagging tape that we were drawing stripes on until last 12 plants of block D. Then began using new tape from Lizzie.
\item If both cordons (or canes) looked fairly similar, we selected which cordon (or cane) to use by flipping a coin
\item An observation is that we saw a heard of horses grazing within the vineyard here. Is this for a land management reason, or an agreement between land owners/horse owners? DO the horses graze there all year? DO they damage or help (fertilize) the vines? 
\item We noticed in block C, there was quite a bit of within block phenological variation. Northern plants of row 30 were bursting bud or 2 leaves, and at the south end of the same row there were 3/4 leaves out
We saw a weather data logger at WHitetail 
\item Mcintyre is on a bench/plato overlooking the western side of the Okenagan Valley. You can just see the tail end of the lake from some places.
\item we saw what we think are fans in both Mcintyre and Whitetail vineyards 
\item Mira noted there were totally different weed communities around the two vineyards. Might this suggest different soil types? We also saw what we think is evidence of fertilization, bluie green pellets in the soil.
\item Lizzie suggested we think about buying a cheap printer, or asking Carl and Pat to print. Worst case Lizzie will post us some. 

\subsubsection {schedule}
7.15 am left Penticton \\
8.00am Arrived at Whitetail \\
We drove from lower vineyard to upper vineyard by following the road past OIB block. The road slopes up the side of the hill. \\
8.10am got to block J \\
Set up block J sampling \\
9.15am Left upper fields \\
9.30am Drove through block middle of OIB \\
We set up 36 vines in block OIB and 36 in block B \\
11.18am Finished setting up blocks OIB and block B, and started driving to block C \\ 
11.30am set up 24 vines in block C Chardonnay. \\
We then went to Sav B. We noticed some vines have overhead and drip irrigation, where as others had only overhead \\
12.20 to 1.20 lunch break, and Lizzie visiting \\
1.23pm  drive to block B Sav B to finish last 12 vines \\
1.30pm finished last 12 vine set up. \\
2.15pm we headed to Merlot block G. Many plants here were too young. Also rows 55 plus were old but had lots of ripped up vines. we didnt end up setting anything up here yet \\
2.30pm we popped over to the big metal shed where we met Scot (we think) and he gave us keys to McIntre. We need to return the key at the end of the 3 weeks. 
2.40pm we went back to the vineyard and set 36 vines up at block G Cab Sav \\
3.08 left Whitetail for McIntre vineyard \\
3.20 arrived McIntre \\
3.30 sat and though about stuff and took photos \\
3.33 drove around to block Q and set 12 vines up\\
3.48 drove around \\
3.53 started setting more vines in block Q \\ 
4.04 finished setting the extra vines \\
4.07 drove to Pinot noir block O \\
Block O has a mix of cordon and cane training vines. \\
4.28pm finished the setting 24 vines in block O\\
4.33 pm Stopped at the other end of block O Pinot noir to set up last 12 vines close to the road\\
4.41 pm finished setting up vines and went home \\
5.30 pm arrived back in Penticton \\




\end{smitemize}


\subsection{TO DO:}
\begin{smitemize}
\item Make "how to get where document" add links to Carl's maps on github
\item Seb Farms names - first 2 letters = variety, 3rd letter = block? Add block for each variety to the variety names document for Carl
\item We should ask if we can drive on the Arterra vineyards.
\item Ask Arterra managers if they prefer text or call communication
\item Find not pink, orange, yellow flagging tape - Growers Supply?

\end{smitemize}

\end{document}