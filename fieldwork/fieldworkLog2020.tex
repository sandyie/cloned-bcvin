\documentclass[11pt,letter]{article}
\usepackage[top=1.00in, bottom=1.0in, left=1.1in, right=1.1in]{geometry}
\renewcommand{\baselinestretch}{1.1}
\usepackage{graphicx}
\usepackage{natbib}
\usepackage{amsmath}
\usepackage{textcomp}
\usepackage[
singlelinecheck=false % <-- important
]{caption}
\usepackage{float}
\usepackage{hyperref}

\def\labelitemi{--}
\parindent=0pt

\def\labelitemi{--}
\parindent=0pt

\newenvironment{smitemize}{
\begin{itemize}
  \setlength{\itemsep}{0pt}
  \setlength{\parskip}{0.8pt}
  \setlength{\parsep}{0pt}}
{\end{itemize}
}


\begin{document}
\bibliographystyle{/Users/Lizzie/Documents/EndnoteRelated/Bibtex/styles/besjournals}
\renewcommand{\refname}{\CHead{}}

\title{bcvin Fieldwork Log, May 2020}
\author{}
\maketitle
\tableofcontents

\subsection{4 May 2020}
\begin{smitemize}
\item Met with Pat Bowen and Carl Bogdanoff to discuss sampling and vineyard etiquette.
\item We should be allowed to leave up tape throughout the season and mark buds with metal tags but should ask first, of course
\item May be allowed to tag vines until next year for easy finding - ask managers
\item Pat and Carl don't think Arterra or Quail's Gate are doing organics management. Sebastian Farms is all organic now.
\item Merlot or Chardonnay would be good varieties to sample more throughly because they are commonly grown and popular for the region.
\item Slope aspect, east vs west side of lake (getting morning or afternoon sun), row direction - interests for climatic variation
\item the vineyard managers we are meeting should be considered more collaborators than facilitators. They might have good ideas.
\item to keep vine quality consistent, we should focus on vineyards directly managed by the big wine companies rather than contract growers. This is because contract growers get paid based on the amount of grapes they sell, rather than the quality of the grapes, so there is a tendency to overcrop. Mike may suggest a contract grower though that is growing to the higher standards, and if so we should go with his suggestion. As long as we keep to high quality vine management, the wine company should not make a big difference.
\item the most popular grape varieties shifts regularly. At the moment there is a lot of shifting to merlot from white winegrapes in the south. How it is probably 70:30 ratio of red to white there. More centrally in the valley it is more like 50:50. Sav blanc is getting more popular, and pinot gris is also popular at the moment
\item vines should be at least 4 years old for us to monitor them

\end{smitemize}

Grower communication
\begin{smitemize}
\item send growers information in easily digestible and usable formats i.e. simple spreadsheets or maps
\item make yourself useful by giving growers information that they want
\item Try never to say no to a talk or committee meeting invitation
\item Get on the Research and Development Council for BC Wine. Maybe attend conference if there is one.
\item when writing grants, deliverables must be useful.

\end{smitemize}

Need to consider: Make sure vines we choose will be there for 3 years
\vspace{1.5ex}

 We also briefly discussed winter hardiness modeling.
 \begin{smitemize}
\item Carl has started using a 4th order Quadratic equation, and this seems to be working well
\item acclimation and deacclimation are not the same process, but Carl thinks they should have similar rates
\item Carl also doesn't think there should be a relationship between maximum hardiness and rate of deacclimation. He pointed out Riesling bursts quite late, whereas Chardonnay bursts early, but they are both fairly hardy

\end{smitemize}

\subsection{5 May 2020}
\begin{smitemize}
\item met with Mike Watson at Dark Horse Vineyard, vineyard manager?/head viticulturist? at Arterra to discuss site access and what to consider when choosing vines
\item Mike is interested in soil type. Soil type can differ a within blocks, although less so for Arterra because Mike tries to avoid this issue. For contract growers, especially new growers, it is more of a problem because they focus on maximizing yield per hectare, so squish in as many vines as possible
\item Mike also mentioned something about a contract grower who "pushed the vines too much", who 3rd leaf on one year and second leaf the next, but mostly got away with it, although there were "circles of death" at the site. Mike seemed to think this might be related to the soil. He said some growers can get away with doing everything wrong, while others cant, and maybe this is because of soil. We didn't understand this section of the conversation.
\item some site maps might be incorrect, so Mike will send us updated maps
\item Fine to sample at Dark Horse, Whitetail, McIntrye, NK'MIP Cellars.
\item Dark Horse - We need to let Mike know we will be around. If we keep things non-onerous they can collect some data for us.
\item No expected variety/plant changes in the any of vineyards.
\item Flagging is fine, and we don't need signs as we don't want the workers to do anything special. We should use flagging tape that is not orange, yellow or pink though. (Faith also saw some blue while out at the vineyards).
\item Be aware that Inkameep Winery and NK'MIP Cellars are different. We are working at NK'MIP Cellars.
\item Phone numbers (to be contacted the day before entering the vineyard)
\item Manjit Deol (Manager at McIntrye and Whitetail) - 250 488 8215
\item Scott Carlson (Vineyard supervisor at McIntrye and WHitatail) - 250 485 7920. Call preferentially to Manjit because on site more
\item NK'MIP - Nelson Dutra (not sure of surname spelling) - 250 485 8085
\item there are also boards up at each vineyard with the spraying schedule. If in doubt, call Mike.
\item Call the day before to let vineyard know we will be sampling.
\item NK'MIP is usually locked, but we have a key. We should return the key at the end of the season.
\item we should access the vineyards between 6am and 3/4 pm. That is when the workers are there.
\item weekends should be ok. The workers work Saturdays but not Sundays though, so maybe we should avoid Sundays?
\item Nothing is alarmed
\item Mike talked very enthusiastically about a GIS dashboard they are working on that will have spraying schedules and stuff. Faith wonders if we could make our information (when we have it) a GIS dashboard, or a layer for their dashboard?
\item when we asked Mike what phenological stages he looks for, he mentioned "when the inflorescence separates". He uses this to gage what pesticides to use (whether to combine sulfur and pertritas or not). He's not sure how accurate this stage is, and we are not totally sure which Eichhorn - Lorenz stage it is, but we should try and capture it. It's a bit later than we would have thought to measure.

\end{smitemize}

Other notes from our Adventures:
\begin{smitemize}
\item why do Arterra cane prune rather than cordon prune some of the vines? Also we should consider how our flagging will work with these pruning types. Probably we need to use loose flagging so we don't restrict cane growth
\item Why do they spray irrigate in Whitetail? Also avoid rows where there are irrigation issues.
\item Mike has friends who rent Apex condos during summer if we can't find housing
\item Shiraz is Syrah! New revelation for Mira!
\item NK'MIP Cellars and McIntrye have Shiraz/Syrah
\item All 4 Arterra vineyards have Cabernet Sauvignon - consider adding to variety list

\end{smitemize}

Dark Horse Vineyard Walk notes
\begin{smitemize}
\item We should have 6 vines next to each other as a unit, and always skip to the first pole to avoid vines right on the outside edge of the block
\item We want probably a ratio of 40:40:20 of upper:lower:middle vines in the row. We also want to be as efficient with our time as possible, so avoid unnecessary crossing vine rows.
\item Sample near weather station
\item Merlot and Chardonnay are across the road from each other so make sampling plan easy to grab both.
\item Sample Merlot from several blocks and Chardonnay from both blocks

\end{smitemize}

NK'MIP Vineyard Walk notes
\begin{smitemize}
\item Sample both Merlot blocks, both Shiraz blocks, Cab Sau?
\item 6 top, 6 bottom, a few in middle

\end{smitemize}


\subsection{6 May 2020}

Chad meeting at Quail's Gate Wineshop to discuss Quail's Gate Estate Vineyard and Mannhardt Vineyard
\begin{smitemize}
\item Syrah/shiraz at Quail's Gate is not doing very well. In fact very few of the vines below Boucherie Rd are doing very well. They are likely to be removed. The sav blanc is ok though. Chad thinks this might be because the soil down there is more clay, whereas the top is more volcanic soil and gravel  
\item Chad was  not sure if there were good soil maps (we asked Pat and Carl and they said they have maps of soil)

\end{smitemize}



\subsection{TO DO:}
\begin{smitemize}
\item Make "how to get where document" add links to Carl's maps on github
\item Seb Farms names - first 2 letters = variety, 3rd letter = block? Add block for each variety to the variety names document for Carl
\item We should ask if we can drive on the Arterra vineyards.
\item Ask Arterra managers if they prefer text or call communication
\item Find not pink, orange, yellow flagging tape - Growers Supply?

\end{smitemize}

\end{document}